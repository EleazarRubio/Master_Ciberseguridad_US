%!TEX root =../MemoriaTFM.tex
%El anterior comando permite compilar este documento llamando al documento raíz
\chapter{Descripción de la Técnica}\label{chp-02}
\epigraph{A good DevOps organization will free up developers to focus on doing what they do best: write software. }{Rob Steward, 2015\\Global Vicepresident at Verint-Systems.}

\lettrine[lraise=-0.1, lines=2, loversize=0.2]{P}{ara} comprender el desarrollo del trabajo aquí presentado, tal y como se ha llevado a cabo, se debe conocer la situación en que éste se desarrolla, la tecnología de la que se dispone y los elementos existentes y necesarios, de una manera objetiva.

Es por esto, que el apartado actual está orientado a conocer las características de la realidad representada y a introducir las bases tecnológicas del presente \gls{TFM}, resaltando los conceptos más importantes.

\section{Empresas \gls{TI}}

\TODO{Pipelines y demás en empresas modernas... ¿Cómo se hacen las cosas? y comentar de qué forma no se va a interferir de manera destructiva en este proceso... y quizás esto último en otro apartado... ¿Incluyo aquí cómo puede ser "más o menos" un día de trabajo DevOps?}

\section{Integración Continua \gls{IC}}

(Continuous Integration, \gls{CI})
(Software de) 

\section{CVE}

\TODO{Breve muy breve}

\section{Análisis estático}



\subsection{Ruby y NodeJS}

\TODO{YAML no e stá por ningún lado.}

Quizás a estas alturas no sea necesario hablar de estos lenguajes de programación... lo que si que daré será algunos datos que confirmen un poco por qué he empezado con estos...

\TODO{Breve presentación a la importancia de estos lenguajes de programación, comentando que lo que se ha hecho aquí es extensible a otros lenguajes, con otras herramientas similares.\\Concepto dependencias de código.}

Bundler is the de facto way of managing dependencies. It provides, among other things, a clear way of specifying required libraries and their versions, by keeping track of everything for you through Gemfile and (for applications) Gemfile.lock. Exactly the sort of information you’d need when checking for security vulnerabilities.

\subsection{De las dependencias del código}

\TODO{¿Qué es y cómo funciona?}

\subsection{De contenedores de imágenes}

\TODO{Quizás no son necesarios los subapartados... y simplemente baste con explicar elconcepto de análisis estático de algo genérico.}

https://www.linuxadictos.com/docker-i-que-es-conociendo-la-ballena.html

http://www.javiergarzas.com/2015/07/que-es-docker-sencillo.html

https://www.redeszone.net/2016/02/24/docker-funciona-la-virtualizacion-contenedores/


La nube es cada vez más grande, más potente, cuenta con más usuarios que hacen uso de ella al mismo tiempo y, además, permite la ejecución de aplicaciones cada vez más potentes, por lo que, para garantizar el correcto funcionamiento de esta, tanto en el presente como en el futuro, es necesario utilizar una plataforma que optimice los recursos lo mejor posible y, al mismo tiempo, sea lo más escalable posible con el fin de poder ampliar sus características de forma sencilla cuando sea necesario.

La nube es sinónimo de virtualización. Ejecutar un sistema operativo virtual por cada instancia de una aplicación es un proceso muy pesado y poco optimizado, a la vez que lento. Por ello, la comunidad Linux ha trabajado en el concepto de contenedores, una nueva forma de optimizar recursos creando pequeños espacios virtuales de las aplicaciones necesarias cargando solo el núcleo de la aplicación y las dependencias, pero funcionando siempre sobre un único kernel, o sistema operativo. (Esquema de esta WEB)

\section{Trabajos relacionados}

\endinput