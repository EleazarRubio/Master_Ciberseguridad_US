%!TEX root =../MemoriaTFM.tex
\chapter*{Resumen}
\pagestyle{especial}
\chaptermark{Resumen}
\phantomsection
\addcontentsline{toc}{listasf}{Resumen}

\lettrine[lraise=-0.1, lines=2, loversize=0.2]{E}{n} los últimos tiempos de las empresas dedicadas al desarrollo de \gls{SaaS}, los conceptos de metodología ágil y la filosofía \gls{DevOps} están cobrando, cada vez más, un papel fundamental para el desarrollo de las mismas\cite{consultorit2017}. Según un estudio de alcance mundial realizado recientemente por CA Technologies, más del 75 por ciento de las organizaciones españolas coinciden en que las metodologías ágiles y \gls{DevOps} son cruciales para el éxito de la transformación digital\cite{catechnologies2017}. 

Este nuevo modo de entender el mundo del desarrollo de \gls{SW} posee una serie de elementos comunes, cada uno de ellos implementado con herramientas cada vez más conocidas y populares para las empresas que lo llevan a la práctica: 

\begin{itemize}
	\item Plataformas de desarrollo colaborativo y control de versiones de \gls{SW} (por ejemplo \href{https://github.com/}{GitHub}), donde se almacena el código desarrollado por las mismas.
	\item Diferentes entornos o infraestructuras de trabajo para los desarrolladores, que van a permitir un desarrollo y despliegue continuo para las mejoras del producto: entornos de desarrollo, entornos de seguro de calidad o \gls{QA}, preproducción, producción o entorno final, etc.
	\item Fundamentos de \gls{IC} y \gls{DC}.
	\item Plataformas basadas en el despliegue de contenedores (generalmente \href{https://www.docker.com/}{Docker}) que encapsulan los distintos elementos que componen el producto final y optimizan los recursos utilizados en las máquinas que los contienen, en su mayoría subcontratadas a terceras compañías (\href{https://aws.amazon.com/es/}{Amazon Web Services}, \href{https://azure.microsoft.com/es-es/}{Microsoft Azure}, etc.).
\end{itemize}

El presente \gls{TFM} pretende exponer un proceso automatizado de análisis estático de aplicaciones en varios niveles. El resultado de la ejecución periódica de estos procedimientos genera informes de seguridad que podrán ser utilizados para prevenir amenazas durante las fases más tempranas del \gls{CDS} del software, advirtiendo de aspectos tales como:

\begin{enumerate}
	\item Si la aplicación creada tiene Vulnerabilidades (\href{https://cve.mitre.org/}{\gls{CVE}}) en las librerías de dependencias de código utilizadas.
	\item Si la imagen que se va a emplear para desplegar el contenedor de dicho \gls{SW} contiene vulnerabilidades conocidas al nivel de \gls{SO}.
\end{enumerate}

\TODO{TODO - Cambiar enlaces por referencias bibliográficas. Deployment and orchestration of immutable application images using containers (mostly Docker) that increases the portability, reusability and scalability of the application. These applications use Cloud Providers as the underlaying platform such as AWS, Azure or GCE among others. TRADUCIRLA EN EL PUNTO}

\chapter*{Abstract}
\pagestyle{especial}
\chaptermark{Abstract}
\phantomsection
\addcontentsline{toc}{listasf}{Abstract}

\lettrine[lraise=-0.1, lines=2, loversize=0.2]{L}{ately} inside Software as a Service (\gls{SaaS}) companies, the agile methodology and \gls{DevOps} philosophy concepts are taking a fundamental role for the development of them\cite{consultorit2017}. According to a recent global survey by CA Technologies, more than 75 percent of Spanish organizations agree that DevOps and agile methodologies are crucial to the success of digital transformation\cite{catechnologies2017}.

This new way of understanding the world of \gls{SW} development has various common elements, each of them implemented with well known and popular tools for the companies that put it into practice:

\begin{itemize}
\item Collaborative development platforms and \gls{SW} version control systems (for example \href{https://github.com/}{GitHub}), where the code developed is keeped.
\item Different environments and infrastructures for developers, which will allow to continuous development and deployment for product enhancements: development environments, Quality Assurance \gls{QA} environments, pre-production, production or final environment, etc.
\item \gls{CI} and \gls{CD} principles.
\item Deployment and orchestration of immutable application images using containers (mostly Docker) that increases the portability, reusability and scalability of the application. These applications use Cloud Providers as the underlaying platform such as AWS, Azure or GCE among others.
\end{itemize}

This thesis tries to expose a procedure to automatically implement an application static analysis pipeline at several layers. The outcome of these periodically executed pipelines provide the company security reports that can be used to prevent threats during the earliest stages of the \gls{SDLC}:

\begin{enumerate}
\item If the application created has Vulnerabilities (\href{https://cve.mitre.org/}{\gls{CVE}}) in the code dependencies used.
\item If the image used to deploy the container of the mentioned \gls{SW} contains known vulnerabilities at the \gls{OS} level.
\end{enumerate}

