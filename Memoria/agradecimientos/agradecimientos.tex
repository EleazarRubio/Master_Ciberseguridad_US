%!TEX root =../MemoriaTFM.tex
\chapter*{Agradecimientos}
%\pagestyle{especial}
\pagestyle{empty}
%\chaptermark{Agradecimientos}
\phantomsection
%\addcontentsline{toc}{listasf}{Agradecimientos}
%\vspace{1cm}
%{\huge{Agradecimientos}}
%\vspace{1cm}

\lettrine[lraise=-0.1, lines=2, loversize=0.25]{A}{unque} el trabajo aquí realizado pueda llevar al engaño por su propia definición, Trabajo \textit{Fin} de Máster, nada más lejos de la realidad, ya que esta etapa que ahora parece concluir, para mi no ha sido más que el principio y no ha hecho más que comenzar.

Creo que no desvarío si afirmo (rotundamente) que el Máster que aquí acaba ha sido el comienzo a mi nueva vida: nueva profesión, nueva ciudad, en un nuevo país, nuevos amigos, un número incontable de vuelos de ida y vuelta y un sin fin de nuevas experiencias, emociones y sensaciones que me dieron un fuerte empujón fuera de mi bien establecida \textit{zona de confort} (o así la llaman). 

"Al César lo que es del César" y hasta aquí lo que le pertenece. 

Lo cierto es que todo lo que ha provocado el haberme atrevido a seguir estudiando, después de una carrera tan larga, y todos los cambios y nuevas etapas que ha traído este Máster de la mano no tiene un sólo culpable al que dar las gracias, sino más bien varios.

Gracias a mi familia, a toda al completo, y no son pocos. Tengo la suerte de tener una gran familia que me apoya, me quiere, me comprende y cuida de mi por muy lejos o muy cerca que me encuentre, sabiendo siempre sacar lo mejor de mi. Diría que os debo la vida, pero es obvio.

Gracias a mi novia Eva, por ser como es, sin más. porque su amor infinito y su apoyo incondicional me hacen querer ser mejor persona cada día.

Gracias a mi amigo Pablo, porque si el que tiene un amigo tiene un tesoro, mi tesoro no tiene precio.

Gracias a Sergio, mi nuevo maestro de profesión, porque su paciencia y esfuerzo conmigo no tiene límites y por ser la persona que con ilusión "maquinó" la idea de la que ha surgido este Trabajo Fin Máster.

Y gracias a todas aquellas personas que confiaron en mi, ofreciéndome la oportunidad de crecer personal y profesionalmente, ¡No pienso desaprovecharla!. 

Esto no es más que el principio.

{\flushleft{\hfill \emph{Eleazar Rubio Sorrentino}}}%
\vspace{-.3cm}
{\flushleft{\hfill \emph{Sevilla, 2017}}}%