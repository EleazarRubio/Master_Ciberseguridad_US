%!TEX root =../MemoriaTFM.tex
%El anterior comando permite compilar este documento llamando al documento raíz
\chapter{Conclusiones}\label{chp-05}
\epigraph{Security is not a line in the sand. Protecting your business, customers, citizens’ data, should be always your number one priority}{Dr. Werner Vogels, 2017\\CTO at Amazon.com}

\lettrine[lraise=-0.1, lines=2, loversize=0.2]{E}{l} apartado actual, tras haber recorrido en la memoria de este \gls{TFM} "\eltitulo" toda la información relativa al proyecto, está dedicado a presentar los objetivos alcanzados y a descubrir algunas de las posibles líneas futuras de trabajo en torno al mismo. 

El conjunto de aplicaciones aportadas a las herramientas y entorno utilizados surge ante la necesidad de las empresa moderna de vigilar de una manera no intrusiva los posibles agujeros y vulnerabilidades de seguridad que contengan las aplicaciones desarrolladas por ellas y los sistemas donde son desplegados. Esta tarea viene requiriendo un esfuerzo excesivo de parte de aquellas personas que se preocupan por mantener los sistemas lo más protegidos posible. 

Uno de los focos de atención en este proceso son las vulnerabilidades en el producto ofertado por la compañía, que pueden estar originadas por la multitud de dependencias de código que va a requerir dicha aplicación para su funcionamiento, así como la inmensa cantidad de imágenes de diferente naturaleza que se despliegan en forma de contenedor en los entornos que dan soporte a las aplicaciones que sustentan el negocio.

Siguiendo esta línea comentada, se decide aportar al proceso de Integración Continua de empresas que ofrecen \gls{SaaS} un sencillo mecanismo de despliegue de contenedores con un conjunto de herramientas y utilidades que van a permitir poder realizar tareas de análisis estático de manera pasiva y como parte del proceso de trabajo natural de la compañía, analizando las posibles posibles vulnerabilidades en las imágenes de contenedores docker y en las depencias  de código utilizado, advirtiendo en el caso de que sean encontradas y aportando informes con los que afrontar la solución al problema.

Por este motivo se puede concluir que, tras el desarrollo de los apartados aquí contenidos, una vez estudiadas las herramientas herramientas aportadas, junto a la posibilidad de despliegue con trabajos automatizados y mecanismo de notificación, se cumple con el objetivo perseguido en la realización de este \gls{TFM}.

Sin embargo, como es trivial en todo proceso tecnológico, la solución aquí entregada no es estática e inamovible y se presta a ser mejorada, además de a mantener los resultados obtenidos actualizados en el tiempo.

Muchas de las tareas y líneas de avance que este nuevo reto contempla aún son desconocidas, ya que van a surgir en la utilización por parte de los usuarios de esta solución. Otras, en cambio, pueden ser establecidas desde el momento actual. A continuación se presentan algunas de las tareas que conforman la línea futura de trabajo de esta aplicación, identificadas durante la realización del proyecto y la redacción de esta memoria:

\begin{itemize}
	\item \textbf{Aumentar el número de lenguajes de programación contemplados en los análisis estáticos}: existen múltiples herramientas en la actualidad de características similares a las aquí presentadas que cubren un amplio abanico de lenguajes de programación diferentes. Implementar un conjunto mayor de dichas herramientas como parte del conjunto actual es una notable mejora a lo aquí presentado.
	\item \textbf{Preparación del entorno para el despliegue en producción}: Contemplando el escenario en que las comunicaciones se realicen con contraseñas encriptadas y utilizando mecanismos de certificados SSL/TLS.
	\item \textbf{Ampliar los repositorios alcanzados}: Aumentando el número de parámetros que los trabajos de Jenkins reciben a la entrada se pueden variar la rama de código a la que se le realiza el análisis, el usuario que almacena el repositorio en su cuenta de GitHub o ambos, ampliando considerablemente el número de posibles repositorios objetivos.  
\end{itemize}

El resultado obtenido a partir del presente \gls{TFM} es un producto con utilidad tangible, que ya está siendo utilizado en los sistemas de Workshare Inc., proporcionando valor añadido a la actividad en materias de Seguridad de la Información y las Comunicaciones que allí se realiza. Este aspecto es elemento clave a la hora de diferenciar el proyecto realizado, ya que existen múltiples trabajos realizados que nunca llegan a superar la barrera de estudios teóricos, para convertirse en un producto o herramienta final en completo funcionamiento, lo que sin duda es una enorme satisfacción para su autor.

\endinput
