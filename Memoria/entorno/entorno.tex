%!TEX root =../MemoriaTFM.tex
%El anterior comando permite compilar este documento llamando al documento raíz
\chapter{Entorno de trabajo}\label{chp-03}
\epigraph{Technology is nothing. What's important is that you have a faith in people, that they're basically good and smart, and if you give them tools, they'll do wonderful things with them.}{Steve Jobs, 1994\\Businessman}

\lettrine[lraise=-0.1, lines=2, loversize=0.2]{A}{ntes} de comenzar el proceso de desarrollo de la aplicación es necesario preparar un entorno adecuado de trabajo, es decir, un conjunto de herramientas hardware y software que permitan llevar a cabo el proyecto con la mayor comodidad y precisión posible.

La correcta elección de un entorno de trabajo adecuado es fundamental a la hora de abordar cualquier tipo de proyecto, ya que el éxito o fracaso, o al menos la eficiencia del proceso de desarrollo del mismo, va a depender en gran medida de dicho entorno utilizado.
El apartado actual presenta el entorno de trabajo utilizado para la realización de este \gls{TFM}.

\section{Git y GitHub}

Los sistemas de control de versiones son programas cuyo principal objetivo es controlar los cambios producidos en el desarrollo de cualquier tipo de software, permitiendo conocer el estado actual de un proyecto, las personas que intervinieron en ellos, etc. Además, un buen control de versiones es tarea fundamental para la administración de un proyecto de desarrollo de software en general\cite{alcazar2014}. Git es uno de los sistemas de control de versiones más populares entre los desarrolladores, es gratuito, open source, rápido y eficiente, aunque gran parte su popularidad es debido a GitHub(\autoref{jenkins-logo}), un excelente servicio de alojamiento de repositorios de software que ofrece un amplio conjunto de características de gran utiilidad para el trabajo en equipo.

\begin{figure}[htbp]
	\centering
	\includegraphics[width=0.80\linewidth]
	{entorno/figuras/github.png}
	\caption{Logotipo de GitHub}
	\label{github-logo}
\end{figure}

A continuación se muestran algunas de las características que han llevado a GitHub a ser tan valorado entre los desarrolladores\cite{quintana2015}:

\begin{itemize}
	\item Permite versionar el código, es decir, guardar en determinado momento los cambios realizados sobre un archivo o conjunto de archivos con la oportunidad de tener acceso al historial de cambios al completo, bien para regresar a alguna de las versiones anteriores o bien para poder realizar comparaciones entre ellas.
	\item Gracias a la gran cantidad de repositorios de \gls{SW} públicos que alberga, es posible leer, estudiar y aprender de el código creado por miles de desarrolladores en el mundo, permitiendo incluso la oportunidad de adaptarlos a las necesidades propias de cada desarrollador, sin alterar el original y realizando una copia o fork\footnote{Copia exacta en crudo del repositorio original que podrá ser utilizada como un repositorio git cualquiera} de este.
	\item Tras haber realizado un fork de un proyecto y haber realizado algunos ajuste, introducido alguna mejora o arreglado algún problema que este pudiera contener, es posible integrar los cambios realizados al proyecto original (previa supervisión de su propietario, administrador o alguno de sus colaboradores), por lo que un repositorio puede llegar a ser construido mediante la contribución una gran comunidad de desarrolladores.
	\item GitHub posee un sistema propio de notificaciones con el que poder estar informado de lo que ocurre en torno a un repositorio concreto, ya sea privado a la compañía o público a la comunidad.
	\item 
\end{itemize}



6. Visor de código

GitHub posee un estupendo visor de código mediante el cual, a través del navegador, podremos consultar en cualquier instante el contenido de archivo determinado, con la sintaxis correspondiente a el lenguaje en el que esté escrito. Este navegador es realmente rápido, y gracias a él podremos hacer pequeñas consultas o copiar porciones de código sin necesidad de bajarse todo el repositorio.

7. Mostrar tus conocimientos

Con Github puedes mostrar tus habilidades como desarrollador(a), puesto que es el código escrito en los archivos, donde reposa el resultado del proceso del desarrollo de software. Al compartir tu cuenta de Github con tu potencial empleador o cliente, este podrá ver la calidad del código que escribes a través de los proyectos públicos que estén en tu cuenta. Todos los proyectos que se escriben para ejecutar una idea, aprender un nuevo lenguaje o tecnología son válidos al momento de exhibir tus conocimientos, así que no dudes en publicarlo en tu cuenta. Como complemento a lo anterior, con Github Pages puedes crear incluso una página como esta que te sirva como portafolio, en la cual puedes escribir sobre ti, los conocimientos que posees o poner enlaces de los proyectos en los que has participado.

8. Registro de incidencias

Cada proyecto creado en Github incluye un sistema de seguimiento de problemas, del estilo sistema de tickets, este permite a los miembros de tu equipo (o a cualquier usuario de GitHub si tu repositorio es público) abrir un ticket escribiendo en este los detalles un problema que tenga con tu software o una sugerencia sobre una función que le gustaría que fuera implementada.

9. Compatibilidad

Github es una plataforma web, por tanto es independiente del sistema operativo que utilices, y además Git que es la herramient que si requiere instalación es compatible con todos los sistemas; Linux, OSX y Windows.

10. Precio

Github, es completamente gratis e ilimitado para proyectos públicos, es decir que todos podrán ver el código que estos contienen (aunque tu siempre tendrás el control sobre quien subirá cambios), sin embargo si deseas puedes tener proyectos privados adquiriendo uno de planes que ofrece, los cuales van desde 7 a 50 dólares mensuales, permitiendo crear 5 y 50 repositorios privados respectivamente.



\section{Bundle Audit}



\section{NSP NodeJS}



\section{Clair (CoreOS)}



\section{Docker}



\section{Slack}



\section{Jenkins}

Jenkins (\autoref{jenkins-logo}) es una herramienta autónoma de código abierto que puede utilizarse para automatizar todo tipo de tareas, como la construcción, prueba y despliegue de software. Jenkins puede ser instalado a través de paquetes de sistemas nativos, Docker, o incluso puede ser ejecutado de manera independiente en cualquier máquina con el entorno de ejecución de Java instalado\cite{jenkins2017}.


\begin{figure}[htbp]
	\centering
	\includegraphics[width=0.80\linewidth]
	{entorno/figuras/jenkins.png}
	\caption{\gls{IC} con Jenkins}
	\label{jenkins-logo}
\end{figure}


Jenkins posee, entre otras, las siguientes ventajas:

\begin{itemize}
	\item \textbf{Continuous Integration and Continuous Delivery}: As an extensible automation server, Jenkins can be used as a simple CI server or turned into the continuous delivery hub for any project.
	\item \textbf{Easy installation}: Jenkins is a self-contained Java-based program, ready to run out-of-the-box, with packages for Windows, Mac OS X and other Unix-like operating systems.
	\item \textbf{Easy configuration}: Jenkins can be easily set up and configured via its web interface, which includes on-the-fly error checks and built-in help.
	\item \textbf{Plugins}: With hundreds of plugins in the Update Center, Jenkins integrates with practically every tool in the continuous integration and continuous delivery toolchain. If a plugin does not exist, you can code it and share with the community.
	\item \textbf{Extensible}: Jenkins can be extended via its plugin architecture, providing nearly infinite possibilities for what Jenkins can do.
	\item \textbf{Distributed}: Jenkins can easily distribute work across multiple machines, helping drive builds, tests and deployments across multiple platforms faster.
	\item It is an open source tool with great community support.
	\item It provides continuous integration pipeline support for establishing software development life cycle work flow for your application.
	\item It also provides support for scheduled builds \& automation test execution.
	\item You can configure Jenkins to pull code from a version control server like GitHub, BitBucket etc. whenever a commit is made.
	\item It can execute bash scripts, shell scripts, ANT and Maven Targets.
	\item It can be used to Publish results and send email notifications.
\end{itemize}


\TODO{Pipeline - A user-defined model of a continuous delivery pipeline, for more read the Pipeline chapter in this handbook.\\Debo poner también la imagen de una pipeline de trabajo con verdes y rojos.}

\endinput