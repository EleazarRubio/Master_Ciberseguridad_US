%!TEX root =../MemoriaTFM.tex
%El anterior comando permite compilar este documento llamando al documento raíz
\chapter{Entorno de trabajo}\label{chp-03}
\epigraph{Technology is nothing. What's important is that you have a faith in people, that they're basically good and smart, and if you give them tools, they'll do wonderful things with them.}{Steve Jobs, 1994\\Businessman}

\lettrine[lraise=-0.1, lines=2, loversize=0.2]{A}{ntes} de comenzar el proceso de desarrollo de la aplicación es necesario preparar un entorno adecuado de trabajo, es decir, un conjunto de herramientas hardware y software que permitan llevar a cabo el proyecto con la mayor comodidad y precisión posible.

La correcta elección de un entorno de trabajo adecuado es fundamental a la hora de abordar cualquier tipo de proyecto, ya que el éxito o fracaso, o al menos la eficiencia del proceso de desarrollo del mismo, va a depender en gran medida de dicho entorno utilizado.
El apartado actual presenta el entorno de trabajo utilizado para la realización de este \gls{TFM}.

\section{Git y GitHub}

Los sistemas de control de versiones son programas cuyo principal objetivo es controlar los cambios producidos en el desarrollo de cualquier tipo de software, permitiendo conocer el estado actual de un proyecto, las personas que intervinieron en ellos, etc. Un buen control de versiones es tarea fundamental para la administración de un proyecto de desarrollo de software en general\cite{alcazar2014}. Git es uno de los sistemas de control de versiones más populares entre los desarrolladores, es gratuito, open source, rápido y eficiente, aunque gran parte su popularidad es debido a GitHub(\autoref{jenkins-logo}), un excelente servicio de alojamiento de repositorios de software que ofrece un amplio conjunto de características de gran utiilidad para el trabajo en equipo.

\begin{figure}[htbp]
	\centering
	\includegraphics[width=0.80\linewidth]
	{entorno/figuras/github.png}
	\caption{Logotipo de GitHub}
	\label{github-logo}
\end{figure}

A continuación se muestran algunas de las características que han llevado a GitHub a ser tan valorado entre los desarrolladores\cite{quintana2015}:

\begin{itemize}
	\item Permite versionar el código, es decir, guardar en determinado momento los cambios realizados sobre un archivo o conjunto de archivos con la oportunidad de tener acceso al historial de cambios al completo, bien para regresar a alguna de las versiones anteriores o bien para poder realizar comparaciones entre ellas.
	\item Gracias a la gran cantidad de repositorios de \gls{SW} públicos que alberga, es posible leer, estudiar y aprender de el código creado por miles de desarrolladores en el mundo, permitiendo incluso la oportunidad de adaptarlos a las necesidades propias de cada desarrollador, sin alterar el original y realizando una copia o fork\footnote{Copia exacta en crudo del repositorio original que podrá ser utilizada como un repositorio git cualquiera} de este.
	\item Tras haber realizado un fork de un proyecto y haber realizado algunos ajuste, introducido alguna mejora o arreglado algún problema que este pudiera contener, es posible integrar los cambios realizados al proyecto original (previa supervisión de su propietario, administrador o alguno de sus colaboradores), por lo que un repositorio puede llegar a ser construido mediante la contribución una gran comunidad de desarrolladores.
	\item GitHub posee un sistema propio de notificaciones con el que poder estar informado de lo que ocurre en torno a un repositorio concreto, ya sea privado a la compañía o público a la comunidad.
	\item GitHub trae incorporado un visor de código, mediante el cual (y a través del navegador) es posible consultar el contenido de un archivo determinado, con la sintaxis correspondiente al lenguaje utilizado y sin necesidad de descargar una copia del mismo.
	\item Cada repositorio de \gls{SW} albergado en GitHub cuenta con su propio seguimiento de incidencias, con un elaborado sistema de tickets, de manera tal que cualquier colaborador (o usuario en general) pueda reportar algún problema encontrado en la utilización el código o pueda simplemente sugerir nuevas características para que sean implementadas.
	\item Al ser una plataforma web es totalmente independiente al \gls{SO} utilizado, siendo por otro lado Git compatible con los principales sistemas actuales: Linux, Windows, OSX.
	\item GitHub es gratuito e ilimitado para repositorios de proyectos públicos, sólo aquellos usuarios que deseen mantener proyectos en privado deberán pagar una cuota. 
\end{itemize}

\TODO{TODO - Una frase de cierre al apartado}


\section{bundler-audit}

Como ya fue comentado en el apartado \ref{chp-02} cada aplicación tiene sus dependencia, y estas a su vez pueden contener vulnerabilidades de seguridad. Encontrar las vulnerabilidades de seguridad que presenta una aplicación es una tarea necesaria y tediosa, que de ser obviada no impedirá que la aplicación generada siga siendo ejecutada como si todo estuviera funcionando en perfectas condiciones, pero que ocultará agujeros en la aplicación que podrán ser utilizados en diversa manera por algún usuario malintencionado.

Para cualquier aplicación escrita con Ruby y en ausencia de alguna herramienta automatizada de análisis de vulnerabilidades, el desarrollador del código deberá estar suscrito a cada lista de correo relacionada con anuncios de seguridad de dependencias y realizar un seguimiento exclusivo de vulnerabilidades y actualizaciones de seguridad de cada una de las dependencias incluidas en la aplicación, para cada aplicación en la que participe\cite{prescott2015}.

Por el contrario, todo este proceso puede ser automatizado en Ruby gracias a bundler-audit\cite{bundleaudit2017}del grupo de colaboradores Rubysec, un verificador a nivel de parche para Bundler con las siguientes características:

\begin{itemize}
	\item bundler-audit analiza las vulnerabilidades en las versiones de las gemas contenidas en el archivo de dependencias \textit{Gemfile.lock} de la aplicación.
	\item Analiza las fuentes de dependencias que puedan ser inseguras (http://).
	\item Permite especificar avisos de seguridad que serán ignorados en el análisis, bien por ser un riesgo asumido por el desarrollador, una vulnerabilidad ya conocida y en la que se está actualmente trabajando o cualquier otro motivo.
	\item No requiere de conexión a internet para cada ejecución que se realice del análisis.
	\item Funcionando cruzando la información de dependencias recogidas del fichero \textit{Gemfile.lock} con una lista de vulnerabilidades conocidas\cite{advisorydb2017}, basada en información pública existente en bases de datos como \gls{CVE}.
\end{itemize}

El código \ref{prg03-01} muestra un ejemplo del resultado obtenido a la salida del terminal de comandos al realizar un análisis estático de vulnerabilidades con bundler-audit al archivo \textit{Gemfile.lock} de un proyecto:

\begin{lstlisting}[language=,caption={Ejemplo de uso de bundler-audit}, breaklines=true, label=prg03-01]
$ bundle audit
Name: actionpack
Version: 3.2.10
Advisory: OSVDB-91452
Criticality: Medium
URL: http://www.osvdb.org/show/osvdb/91452
Title: XSS vulnerability in sanitize_css in Action Pack
Solution: upgrade to ~> 2.3.18, ~> 3.1.12, >= 3.2.13

Name: actionpack
Version: 3.2.10
Advisory: OSVDB-89026
Criticality: High
URL: http://osvdb.org/show/osvdb/89026
Title: Ruby on Rails params_parser.rb Action Pack Type Casting Parameter Parsing Remote Code Execution
Solution: upgrade to ~> 2.3.15, ~> 3.0.19, ~> 3.1.10, >= 3.2.11

Name: activerecord
Version: 3.2.10
Advisory: OSVDB-90072
Criticality: Medium
URL: http://direct.osvdb.org/show/osvdb/90072
Title: Ruby on Rails Active Record attr_protected Method Bypass
Solution: upgrade to ~> 2.3.17, ~> 3.1.11, >= 3.2.12

Name: activerecord
Version: 3.2.10
Advisory: OSVDB-89025
Criticality: High
URL: http://osvdb.org/show/osvdb/89025
Title: Ruby on Rails Active Record JSON Parameter Parsing Query Bypass
Solution: upgrade to ~> 2.3.16, ~> 3.0.19, ~> 3.1.10, >= 3.2.11

Unpatched versions found!
\end{lstlisting}

Además, bundler-audit permite actualizar la base de datos ruby-advisory-db yanalizar el fichero \textit{Gemfile.lock} desde el mismo comando, habilidad que resulta de gran utilidad para su ejecución en sistemas de \gls{IC}, como muestra el código \ref{prg03-02}:

\begin{lstlisting}[language=,caption={Ejecutando bundler-audit tras la actualización de las vulnerabilidades conocidas}, breaklines=true, label=prg03-02]
$ bundle audit check --update
\end{lstlisting}

Por último, y como ya se ha mencionado con anterioridad en este apartado, es posible ignorar advertencias especificadas (código \ref{prg03-03}) por el usuario:

\begin{lstlisting}[language=,caption={Ignorar vulnerabilidades con bundler-audit}, breaklines=true, label=prg03-03]
$ bundle audit check --ignore OSVDB-108664
\end{lstlisting}


\TODO{TODO - Cierre del apartado. de Rubysec para proyectos Ruby}

\section{nsp}

de Node Security Platform para proyectos NodeJS

\section{Clair y Clairctl}

Clair (\autoref{clair-logo}) es un proyecto de código libre desarrollado por CoreOS

\begin{figure}[htbp]
	\centering
	\includegraphics[width=0.80\linewidth]
	{entorno/figuras/clair.png}
	\caption{Clair}
	\label{clair-logo}
\end{figure}


 (CoreOS)

\section{Docker}



\section{Slack}



\section{Jenkins}

Jenkins (\autoref{jenkins-logo}) es una herramienta autónoma de código abierto que puede utilizarse para automatizar todo tipo de tareas, como la construcción, prueba y despliegue de software. Jenkins puede ser instalado a través de paquetes de sistemas nativos, Docker, o incluso puede ser ejecutado de manera independiente en cualquier máquina con el entorno de ejecución de Java instalado\cite{jenkins2017}.


\begin{figure}[htbp]
	\centering
	\includegraphics[width=0.80\linewidth]
	{entorno/figuras/jenkins.png}
	\caption{\gls{IC} con Jenkins}
	\label{jenkins-logo}
\end{figure}


Jenkins posee, entre otras, las siguientes ventajas:

\begin{itemize}
	\item \textbf{Continuous Integration and Continuous Delivery}: As an extensible automation server, Jenkins can be used as a simple CI server or turned into the continuous delivery hub for any project.
	\item \textbf{Easy installation}: Jenkins is a self-contained Java-based program, ready to run out-of-the-box, with packages for Windows, Mac OS X and other Unix-like operating systems.
	\item \textbf{Easy configuration}: Jenkins can be easily set up and configured via its web interface, which includes on-the-fly error checks and built-in help.
	\item \textbf{Plugins}: With hundreds of plugins in the Update Center, Jenkins integrates with practically every tool in the continuous integration and continuous delivery toolchain. If a plugin does not exist, you can code it and share with the community.
	\item \textbf{Extensible}: Jenkins can be extended via its plugin architecture, providing nearly infinite possibilities for what Jenkins can do.
	\item \textbf{Distributed}: Jenkins can easily distribute work across multiple machines, helping drive builds, tests and deployments across multiple platforms faster.
	\item It is an open source tool with great community support.
	\item It provides continuous integration pipeline support for establishing software development life cycle work flow for your application.
	\item It also provides support for scheduled builds \& automation test execution.
	\item You can configure Jenkins to pull code from a version control server like GitHub, BitBucket etc. whenever a commit is made.
	\item It can execute bash scripts, shell scripts, ANT and Maven Targets.
	\item It can be used to Publish results and send email notifications.
\end{itemize}


\TODO{Pipeline - A user-defined model of a continuous delivery pipeline, for more read the Pipeline chapter in this handbook.\\Debo poner también la imagen de una pipeline de trabajo con verdes y rojos.}

\endinput