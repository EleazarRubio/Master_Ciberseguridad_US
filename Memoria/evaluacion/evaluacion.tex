%!TEX root =../MemoriaTFM.tex
%El anterior comando permite compilar este documento llamando al documento raíz
\chapter{Introducción}\label{chp-01}
\epigraph{You can ask 10 people for a definition of DevOps and likely get 10 different answers.}{Dustin Whittle, 2014}

\lettrine[lraise=-0.1, lines=2, loversize=0.2]{E}{n} 






DevOps es una metodología de trabajo basada en el desarrollo de código que usa nuevas herramientas y prácticas para reducir la tradicional distancia entre técnicos de programación y de sistemas. Este nuevo enfoque de colaboración que es DevOps permite a los equipos trabajar de forma más cercana, aportando mayor agilidad al negocio y notables incrementos de productividad.




Hay un pilar tecnológico indiscutible en DevOps que es la automatización. Herramientas tecnológicas como Docker, Puppet, Jenkins o AWS Lambda están tan asociadas a DevOps que es difícil no considerar una u otra para acelerar los procesos de desarrollo. Gracias a que es automático el proceso de organizar, probar y desplegar código desde herramientas de integración, la infraestructura cloud puede seguir los ritmos que requieren las aplicaciones en el mercado actual.



\endinput
