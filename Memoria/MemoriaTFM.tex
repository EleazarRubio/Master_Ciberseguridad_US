%:Clase del documento
\documentclass[fontsize=10pt, Myfinal=false, twoside, numbers=noenddot]{scrbook}
%Minion=true, English=true, Myfinal=true

%:Paquete de estilos original
\usepackage{libroETSI}

%:Paquete específico para cargar tikz (y sus librerías) y pgfplots
\usepackage{dtsc-creafig}

%:Paquete para notaciones específicas
\usepackage{notacion}

%:Paquete para incorporar aspectos concretos de la edición
\usepackage{edicionPFC}

\usepackage{indentfirst}

%:Espacio de una línea entre párrafos
\setlength{\parskip}{\baselineskip}

%:Evitar que LaTex corte las palabras.
\pretolerance=2000
\tolerance=3000

\newcommand{\bibhref}[3][blue]{\href{#2}{\color{#1}{#3}}}%

\AtBeginDocument{%
	\renewcommand\bibname{Referencias}
}

\newcommand\TODO[1]{\textcolor{red}{#1}}
%:\renewcommand\TODO[1]{} %PARA OCULTARLAS

%:Para modificar fácilmente la fuente del texto.
\makeatletter
\ifdtsc@Minion
\ifluatex
\setmainfont[Renderer=Basic, Ligatures=TeX,	% Fuente del texto 
Scale=1.01,
]{Minion Pro}
% En este caso conviene modificar ligeramente el tamaño de las fuentes matemáticas
\DeclareMathSizes{10}{10.5}{7.35}{5.25}
\DeclareMathSizes{10.95}{11.55}{8.08}{5.77}
\DeclareMathSizes{12}{12.6}{8.82}{6.3}
\fi
\else
\ifluatex
\setmainfont[Renderer=Basic, Ligatures=TeX, 
Scale=1.0,
]{Times New Roman}
\else
\usepackage{tgtermes}
\fi
\fi
\makeatother

\makeglossaries

%Lista de acrónimos para el glosario
\newacronym{SaaS}{SaaS}{software como servicio}
\newacronym{DevOps}{DevOps}{Development and Operations}
\newacronym{SW}{SW}{software}
\newacronym{QA}{QA}{Quality Assurance}
\newacronym{CI}{CI}{Continuous Integration}
\newacronym{TFM}{TFM}{Trabajo Fin de Máster}
\newacronym{SO}{SO}{Sistema Operativo}
\newacronym{OS}{OS}{Operative System}
\newacronym{TI}{TI}{Tecnología de la información}
\newacronym{CVE}{CVE}{Common Vulnerabilities and Exposures}
\newacronym{IC}{IC}{Integración continua}
\newacronym{API}{API}{Interfaz de Programación de Aplicaciones}
\newacronym{CD}{CD}{Continuous Deployment}
\newacronym{DC}{DC}{Despliegue Continuo}
\newacronym{SDLC}{SDLC}{Secure Development Lifecycle}
\newacronym{CDS}{CDS}{Ciclo de Desarrollo Seguro}
\newacronym{AWS}{AWS}{Amazon Web Service}
\newacronym{GCE}{GCE}{Google Cloud Engine}

% Formato A4
\geometry
{paperheight=297mm,%
	paperwidth=210mm,%
	top=25mm,%
	headsep=8.5mm,%
	includefoot, 
	textheight=240mm, 
	textwidth=150mm, 
	bindingoffset=0mm, 
	twoside}

\usepackage[a4,center]{crop}%para poner las cruces de esquina de página, poner la opción cross

%:Esquema de numeración por defecto
\setenumerate[1]{label=\normalfont\bfseries{\arabic*.}, leftmargin=*, labelindent=\parindent}
\setenumerate[2]{label=\normalfont\bfseries{\alph*}), leftmargin=*}
\setenumerate[3]{label=\normalfont\bfseries{\roman*.}, leftmargin=*}
\setlist{itemsep=.1em}
\setlength{\parindent}{1.0 em}

% El nivel hasta el que se muestra el índice 
\setcounter{tocdepth}{1}

%:Empieza el documento

\begin{document}
	
	%:Para incluir toda la referencia bibliográfica aunque no se cite, descomente la siguiente línea
	%:\nocite{*}
	
	%:Inicio de la portada
	
	%:Para crear la portada y la portada interior (pagina titular)
	\titulo{Seguridad en la integración continua de la metodología ágil y la filosofía DevOps} %\mbox evita que se divida una palabra al cambiar de línea
	\autor{Eleazar Rubio Sorrentino}
	\director{Juan Manuel Vozmediano Torres}
	\titulodirector{Profesor Titular}
	
	\departamento{Dep. de Ingeniería Telemática}
	\centro{Escuela Técnica Superior de Ingeniería}
	\universidad{Universidad de Sevilla}
	\titulacion{Ingeniería de Telecomunicación}
	\fecha{2017}
	\nombretrabajo{Trabajo Fin de Máster}
	
	\hypersetup
	{
		linkcolor=black, %Enlaces en color negro.
		pdfauthor={\elautor},
		pdftitle={\nombretrabajo,\eltitulo}, 
		pdfkeywords={Latex, edición, formato de texto}	
	}
	
	%:logo de la Universidad y logo del departamento.
	\portadaPFC{figuras/LogoUS.pdf}{figuras/LogoIT.pdf}
	
	%:Fin de la portada
	
	%:Todo lo que constituye la primera parte del libro que no es el cuerpo del libro en realidad
	\frontmatter
	%:Pone la numeración en mayúscula (al menos en español).
	\pagenumbering{Roman}
	
	%!TEX root =../MemoriaTFM.tex
\chapter*{Agradecimientos}
%\pagestyle{especial}
\pagestyle{empty}
%\chaptermark{Agradecimientos}
\phantomsection
%\addcontentsline{toc}{listasf}{Agradecimientos}
%\vspace{1cm}
%{\huge{Agradecimientos}}
%\vspace{1cm}

\lettrine[lraise=-0.1, lines=2, loversize=0.25]{A}{partado} de agradecimientos.




{\flushleft{\hfill \emph{Eleazar Rubio Sorrentino}}}%
\vspace{-.3cm}
{\flushleft{\hfill \emph{Sevilla, 2017}}}%
	
	%!TEX root =../MemoriaTFM.tex
\chapter*{Resumen}
\pagestyle{especial}
\chaptermark{Resumen}
\phantomsection
\addcontentsline{toc}{listasf}{Resumen}

\lettrine[lraise=-0.1, lines=2, loversize=0.2]{E}{n} los últimos tiempos de las empresas dedicadas al desarrollo de \gls{SaaS}, los conceptos de metodología ágil y la filosofía \gls{DevOps} están cobrando, cada vez más, un papel fundamental para el desarrollo de las mismas\cite{consultorit2017}. Según un estudio de alcance mundial realizado recientemente por CA Technologies, más del 75 por ciento de las organizaciones españolas coinciden en que las metodologías ágiles y \gls{DevOps} son cruciales para el éxito de la transformación digital\cite{catechnologies2017}.

Este nuevo modo de entender el mundo del desarrollo de \gls{SW} posee una serie de elementos comunes, cada uno de ellos implementado con herramientas cada vez más conocidas y populares para las empresas que lo llevan a la práctica: 

\begin{itemize}
	\item Plataformas de desarrollo colaborativo y control de versiones de \gls{SW} (por ejemplo GitHub\cite{github2017}), donde se almacena el código desarrollado por las mismas.
	\item Diferentes entornos o infraestructuras de trabajo para los desarrolladores, que van a permitir un desarrollo y despliegue continuo para las mejoras del producto: entornos de desarrollo, entornos de seguro de calidad o \gls{QA}, preproducción, producción o entorno final, etc.
	\item Fundamentos de \gls{IC} y \gls{DC}.
	\item Desarrollo de aplicaciones inmutables que utilizan contenedores de imágenes (generalmente de Docker\cite{docker2017}) para incrementar la protabilidad, reusabilidad y escalabilidad de la aplicación. Estas aplicaciones suelen mantener sus plataformas soportadas en Proveedores Cloud tales como \gls{AWS}\cite{aws2017}, Microsoft Azure\cite{azure2017} o \gls{GCE}\cite{google2017}.
\end{itemize}

El presente \gls{TFM} pretende exponer un proceso automatizado de análisis estático de aplicaciones en varios niveles. El resultado de la ejecución periódica de estos procedimientos genera informes de seguridad que podrán ser utilizados para prevenir amenazas durante las fases más tempranas del \gls{CDS} del software, advirtiendo de aspectos tales como:

\begin{enumerate}
	\item Si la aplicación creada tiene Vulnerabilidades (\gls{CVE}\cite{cve2017}) en las librerías de dependencias de código utilizadas.
	\item Si la imagen que se va a emplear para desplegar el contenedor de dicho \gls{SW} contiene vulnerabilidades conocidas al nivel de \gls{SO}.
\end{enumerate}

\chapter*{Abstract}
\pagestyle{especial}
\chaptermark{Abstract}
\phantomsection
\addcontentsline{toc}{listasf}{Abstract}

\lettrine[lraise=-0.1, lines=2, loversize=0.2]{L}{ately} inside Software as a Service (\gls{SaaS}) companies, the agile methodology and \gls{DevOps} philosophy concepts are taking a fundamental role for the development of them\cite{consultorit2017}. According to a recent global survey by CA Technologies, more than 75 percent of Spanish organizations agree that DevOps and agile methodologies are crucial to the success of digital transformation\cite{catechnologies2017}.

This new way of understanding the world of \gls{SW} development has various common elements, each of them implemented with well known and popular tools for the companies that put it into practice:

\begin{itemize}
\item Collaborative development platforms and \gls{SW} version control systems (for example GitHub\cite{github2017}), where the code developed is keeped.
\item Different environments and infrastructures for developers, which will allow to continuous development and deployment for product enhancements: development environments, Quality Assurance \gls{QA} environments, pre-production, production or final environment, etc.
\item \gls{CI} and \gls{CD} principles.
\item Deployment and orchestration of immutable application images using containers (mostly Docker) that increases the portability, reusability and scalability of the application. These applications use Cloud Providers as the underlaying platform such as \gls{AWS}\cite{aws2017}, Microsoft Azure\cite{azure2017} or \gls{GCE}\cite{google2017}.
\end{itemize}

This thesis tries to expose a procedure to automatically implement an application static analysis pipeline at several layers. The outcome of these periodically executed pipelines provide the company security reports that can be used to prevent threats during the earliest stages of the \gls{SDLC}:

\begin{enumerate}
\item If the application created has Vulnerabilities (\gls{CVE}\cite{cve2017}) in the code dependencies used.
\item If the image used to deploy the container of the mentioned \gls{SW} contains known vulnerabilities at the \gls{OS} level.
\end{enumerate}


	
	%:Índice
	\cleardoublepage
	\phantomsection
	\pagestyle{especial}
	\tableofcontents
	
	%:Empieza el contenido del libro
	\mainmatter
	
	%:Página por defecto
	\pagestyle{esitscCD}
	
	%:Los diferentes capítulos, en carpetas separadas
	
	%!TEX root =../MemoriaTFM.tex
%El anterior comando permite compilar este documento llamando al documento raíz
\chapter{Introducción}\label{chp-01}
\epigraph{You can ask 10 people for a definition of DevOps and likely get 10 different answers.}{Dustin Whittle, 2014\\Developer Advocate at Uber Developer Platform}

\lettrine[lraise=-0.1, lines=2, loversize=0.2]{E}{n} este primer apartado de la memoria se pretende realizar con el lector un recorrido a través del contexto que ha motivado el desarrollo del presente \gls{TFM} para el Máster en Seguridad de la Información y las Comunicaciones  de la Universidad de Sevilla, con el fin de aclarar el objetivo perseguido en su realización. Como conclusión al mismo, se definirá la estructura seguida durante la redacción, introduciendo cada uno de los apartados que se encontrarán a continuación. 

\section{Contexto y motivación}

El año 2017, para empresas englobadas en todo tipo de sectores, está siendo el año de la transformación digital. Estos procesos de transformación exigen distintas formas de trabajo, más ágiles y colaborativas, con las que poder aplicar nuevas tecnologías que permitan conseguir los objetivos del negocio, en entornos que afrontan grandes desafíos culturales, organizativos y operativos e incluso pueden llegar a tener que lidiar con sistemas tecnológicos antiguos y casi obsoletos\cite{expansion2017}. 

Es en este contexto donde el concepto \gls{DevOps} empieza a sonar con más fuerza: el contexto de las metodologías ágiles. 

De esta forma, \gls{DevOps} es un concepto de trabajo, basada en el desarrollo de código, que usa nuevas herramientas y prácticas para reducir la tradicional distancia entre técnicos de programación y de sistemas, respondiendo a la necesidad experimentada por el sector tecnológico de dar una respuesta más rápida a la implementación y operación de aplicaciones. Este nuevo enfoque de colaboración que es \gls{DevOps} permite a los equipos trabajar de forma más cercana, aportando mayor agilidad al negocio y notables incrementos de productividad.

Desde las pruebas de concepto hasta el lanzamiento, pasando por el \textit{testing} y los entornos de prueba, todos los pasos involucrados requieren de la máxima agilidad posible (\autoref{proceso-DevOps}), y eso pasa por integrar los procesos y los equipos de programación con los de sistemas\cite{claranet2017}.

\begin{figure}[htbp]
	\centering
	\includegraphics[width=0.80\linewidth]
	{introduccion/figuras/proceso-devops.png}
	\caption{Introducción al proceso DevOps}
	\label{proceso-DevOps}
\end{figure}

Por otro lado, el concepto de contenedor de aplicación (aislamiento de espacio de nombres y gobernanza de recursos) a pesar de no ser un concepto novedoso, está cobrando cada vez más y más relevancia en el panorama de la empresa actual, de la mano de las continuas mejoras que experimentan las tecnologías que lo implementan, simplificando la administración y transformando la forma en que se desarrolla, distribuye y ejecuta el software, en forma de microservicio\footnote{Aproximación para el desarrollo software que consiste en construir una aplicación como un conjunto de pequeños servicios, los cuales se ejecutan en su propio proceso y se comunican con mecanismos ligeros (normalmente una API de recursos HTTP)}, además de proveer la habilidad de encapsular todo el entorno utilizado con el objetivo de ser desplegado en los sistemas de producción de la empresa, manteniendo las mismas características, aumentando la escalabilidad y disminuyendo notablemente los costes asociados a infraestructuras.

Los contenedores juegan un papel clave en un entorno \gls{DevOps} porque soportan las implementaciones de la pila de desarrollo y operaciones completa y van en camino de formar parte de la definición básica de lo que se conocerá como \gls{DevOps} en unos pocos años\cite{searchdatacenter2015}.

Además, la metodología	 \gls{DevOps} representa una gran promesa a la hora de asegurar el desarrollo del software, ya que las organizaciones pueden potencialmente encontrar y remediar las vulnerabilidades con mayor frecuencia y al principio del ciclo de vida de la aplicación, ahorrando costes y tiempo. Conforme al informe de  \textit{"Seguridad de Aplicaciones y DevOps"} de octubre de 2016 promovido por Hewlett Packard Enterprise\cite{hpe2016}, que incluye tanto respuestas cualitativas como cuantitativas de profesionales de operaciones informáticas, líderes de seguridad y desarrolladores, se concluye que el 99\% de los encuestados confirma que la adopción de la cultura \gls{DevOps} aporta la oportunidad de mejorar la seguridad de las aplicaciones. Sin embargo, solo el 20\% realizan análisis de seguridad de aplicaciones durante el desarrollo y el 17\% no utilizan ninguna tecnología que proteja sus aplicaciones, destacando una desconexión significativa entre la percepción y la realidad de la seguridad \gls{DevOps}.

Es en el contexto planteado donde surge la idea del presente \gls{TFM}: aportar mecanismos a la metodología \gls{DevOps} que permitan analizar la seguridad de las aplicaciones desarrolladas y el contenedor que las albergará dentro de la infraestructura de la empresa, sin interferir de manera destructiva con el propio proceso de desarrollo y depliegue de la aplicación.

\section{Objetivo}

El objetivo del presente Trabajo Fin de Máster (\gls{TFM}) es desarrollar un entorno, basado en contenedores, que pueda ser incluido en el proceso de desarrollo e integración continua de la empresa y con el que poder realizar tareas periódicas programadas para analizar estáticamente las posibles vulnerabilidades (\gls{CVE} entre otras) contenidas en las dependencias de aplicaciones desarrolladas mediante los lenguajes de programación Ruby y NodeJS, además de analizar a nivel del \gls{SO} vulnerabilidades presentes en las imágenes que van a constituir el contenedor que dará soporte a dichas aplicaciones.

Como medio para alcanzar el objetivo planteado se va a hacer uso de una serie de aplicaciones y herramientas, entres las que cabe destacar las siguientes, que serán desarrolladas en los próximos apartados:

\begin{itemize}
	\item GitHub\cite{github2017}: Plataforma de desarrollo colaborativo y control de versiones de \gls{SW} donde almacenar, entre otras, el código desarrollado y que será analizadp.
	\item Jenkins\cite{jenkins2017}: Software de Integración Continua (\gls{IC}) con el que automatizar los trabajos periódicos de análisis estático a realizar.
	\item Docker\cite{docker2017}: Proyecto de código abierto que automatiza el despliegue de aplicaciones dentro de contenedores de software, con el que desplegar los distintos elementos requeridos.
\end{itemize}

El trabajo aquí presentado no tiene como objetivo innovar en la tecnología existente, sino por contra valerse de esta para aportar al futuro usuario una herramienta intuitiva y de fácil aplicación, resultado de la agrupación de otras utilidades, con la que poder desplegar con el mínimo esfuerzo el entorno aquí recopilado. 

\section{Estructura de la memoria}


Para facilitar la lectura de la memoria actual, se cree conveniente presentar un resumen de cómo se estructuran los diferentes apartados que contiene.

En el apartado actual, Introducción, se presenta el \gls{TFM} que se va a realizar, aclarando el objetivo perseguido, el contexto en que surge y los aspectos que motivaron su realización.

El apartado \ref{chp-02}, Descripción de la técnica, se pretende dar a conocer, de manera objetiva, las características de la realidad representada, con rasgos tales como elementos que la componen, utilidad, etc. En concreto, el apartado comienza describiendo repasando el panorama actual en las empresas de software. A continuación se ...

El apartado \ref{chp-03}, Entorno de trabajo, está dedicado a conocer las herramientas utilizadas para poder llevar a cabo el desarrollo y la implementación de la parte técnica de este \gls{TFM}.

El apartado \ref{chp-04}, titulado Desarrollo de la aplicación, es el apartado principal de la memoria. En él, se detalla el proceso a seguir durante el desarrollo e implementación de los objetivos presentados en este proyecto, comenzando preparación del entorno que se va a utilizar en el desarrollo, hasta llegar a presentar el Resultado final obtenido.

Por último, el apartado \ref{chp-05}, está dedicado a las Conclusiones y evaluaciones surgidas del proyecto, los objetivos alcanzados y las líneas futuras de trabajo surgidas durante la realización de éste.

\endinput

	
	%!TEX root =../MemoriaTFM.tex
%El anterior comando permite compilar este documento llamando al documento raíz
\chapter{Descripción de la Técnica}\label{chp-02}
\epigraph{A good DevOps organization will free up developers to focus on doing what they do best: write software. }{Rob Steward, 2015\\Global Vicepresident at Verint-Systems}

\lettrine[lraise=-0.1, lines=2, loversize=0.2]{P}{ara} comprender el desarrollo del trabajo aquí presentado, tal y como se ha llevado a cabo, se debe conocer la situación en que éste se desarrolla, la tecnología de la que se dispone y los elementos existentes y necesarios, de una manera objetiva.

Es por esto, que el apartado actual está orientado a conocer las características de la realidad representada y a introducir las bases tecnológicas del presente \gls{TFM}, resaltando los conceptos más importantes.

\section{Metodología Ágil}

La tecnología actual avanza a una velocidad considerable, provocando a su paso la renovación de la gestión de proyectos informáticos, debiendo esta alcanzar la velocidad de los cambios ocasionados por esta aceleración. 

Así, la calidad, eficiencia, rapidez y flexibilidad en la entrega de un determinado producto se ha convertido en prioritaria, dando paso a la conocida como \textbf{Metodología Ágil}.

La Metodología Ágil envuelve un enfoque para la toma de decisiones en los proyectos software que plantean métodos de ingeniería del software basado en el desarrollo iterativo e incremental, donde los y  soluciones evolucionan con el tiempo según la necesidad del proyecto. Así el trabajo es realizado mediante la colaboración de equipos auto-organizados y multidisciplinarios, inmersos en un proceso compartido de toma de decisiones a corto plazo\cite{vera2014}.

La \autoref{agil} muestra el ciclo de vida en cada iteracción de esta metodología.

\begin{figure}[htbp]
	\centering
	\includegraphics[width=1.0\linewidth]
	{tecnica/figuras/agil.png}
	\caption{CAMBIAR LA IMAGEN}
	\label{agil}
\end{figure}

El uso de procesos ágiles reporta los siguientes beneficios:

\begin{itemize}
	\item Flexibilidad en el proceso y las definiciones de los productos.
	\item Realimentación continua con el cliente.
	\item Iteracción constante del producto, que se va analizando a medida avanza.
	\item Calidad mejorada.
\end{itemize}


\section{Integración Continua \gls{IC} y Despliegue Continuo \gls{DC}}

Integración continua \gls{IC} es una práctica de desarrollo que requiere que los desarrolladores integren nuevos cambios en el código de la aplicación varias veces en un sólo día. Cada vez que esto ocurre, la inserción es verificada por una compilación automática, permitiendo a los equipos de trabajo implicados en el proceso detectar cualquier problema de forma temprana.

Integrando código regularmente los errores pueden ser detectados rápidamente y corregidos con más facilidad, debido a que al tratarse de un proceso frecuente la búsqueda del error va a quedar muy acotada.

El Despliegue Continuo \gls{DC} está estrechamente relacionado a la \gls{IC} y está referido a la liberación en los entornos de producción de la compañía de \gls{SW} que está continuamente siendo probado de manera automática.

Adoptar ambos conceptos (\gls{IC} y \gls{DC}) no solo reduce los riesgos y permite una localización temprana de fallos de código, sino que también permite aumentar la velocidad de trabajo con el \gls{SW}\cite{IC2017}.

\section{Ciclo de Desarrollo Seguro de software (\gls{SDLC})}

Un ciclo de Desarrollo Seguro de software (en inglés, Software Development Life Cycle \gls{SDLC}) es un marco de trabajo que define el proceso utilizado por las compañías a la hora de construir una aplicación desde sus inicios hasta el desmantelamiento de la misma. A lo largo de los últimos años,han surgido multitud de modelos para \gls{SDLC}, que han sido utilizados de diversas maneras acorde a las circunstancias de cada aplicación o empresa en general. Las fases comunes a todos estos modelos para el Ciclo de Desarrollo de software son las siguientes:

\begin{itemize}
	\item Requisitos y planificación.
	\item Arquitectura y diseño.
	\item Planificación de las pruebas.
	\item Desarrollo del código.
	\item Pruebas y resultados.
	\item Lanzamiento y mantenimiento.
\end{itemize}

Con esto, un proceso de implementación de seguridad en el proceso \gls{SDLC} garantizará que las actividades de seguridad, como las pruebas de penetración, la revisión del código y el análisis de la arquitectura, son parte integral del esfuerzo de desarrollo.

\section{Análisis estático}

Análisis estático, o también conocido por análisis estático de código, es un método de depuración de aplicaciones que se basa en exámenes al código cuando éste no se está ejecutando. El proceso de análisis estático es realizado con la ayuda de herramientas automatizadas y asiste a los desarrolladores aportando variada información valiosa mediante el escrutinio del código, utilizando mecanismos y herramientas (por ejemplo revisión de impresiones y campos de formularios) que podrían escapar al ojo humano. El análisis realizado por un humano recibe el nombre de comprensión de código\cite{rouse2017}.

\TODO{Un último párrafo donde enlazar el análisis estático con que se va a realizar de las dependencias del código y con contenedores de imágenes}


Bundler is the de facto way of managing dependencies. It provides, among other things, a clear way of specifying required libraries and their versions, by keeping track of everything for you through Gemfile and (for applications) Gemfile.lock. Exactly the sort of information you’d need when checking for security vulnerabilities.

\section{Virtualización basada en contenedores}

La nube es cada vez más grande, más potente y posee más usuarios. Al mismo tiempo, permite la ejecución de aplicaciones más y más potentes, que requieren garantizar el correcto funcionamiento de esta, actualmente y en el futuro.

Por este motivo, es primordial utilizar ua plataforma que los recursos en la medida de lo posible a la vez que permita la escalabilidad, con el fin de poder ampliar sus características de forma sencilla en caso de ser necesario.

Al hablar de nube se está hablando de virtualización. Ejecutar un \gls{SO} virtual para cada instancia de una aplicación es un proceso muy pesado y lento, es de este problema donde surge el concepto de virtualización basada en contenedores, también conocida como virtualización a nivel de sistema operativo.

Un contenedor no es más que una nueva forma de optimizar los recursos de los que dispone una plataforma creando pequeños espacios virtuales de as aplicaciones necesarias, en las que unicamente será necesario cargar el núcleo de la aplicación y las dependencias de esta, pero funcionando siempre sobre un único kernel, o sitema operativo\cite{velazco2016}. La \autoref{contenedores} muestra las diferencias en el modelo de capas de un sistema de virtualización y tradicional y uno basado en contenedores.

\begin{figure}[htbp]
	\centering
	\includegraphics[width=0.8\linewidth]
	{tecnica/figuras/Contenedores.png}
	\caption{Sistemas de vitualización}
	\label{contenedores}
\end{figure}


\TODO{EN EL APARTADO 3 DIGO QUE EN EL APARTADO 2 EXPLIQUE LAS DEPENDENCIAS Y DEMAS... ASI QUE HAY QUE EXPLICARLAS}

\endinput
	
	%!TEX root =../MemoriaTFM.tex
%El anterior comando permite compilar este documento llamando al documento raíz
\chapter{Entorno de trabajo}\label{chp-03}
\epigraph{Technology is nothing. What's important is that you have a faith in people, that they're basically good and smart, and if you give them tools, they'll do wonderful things with them.}{Steve Jobs, 1994\\Businessman}

\lettrine[lraise=-0.1, lines=2, loversize=0.2]{A}{ntes} de comenzar el proceso de desarrollo de la aplicación es necesario preparar un entorno adecuado de trabajo, es decir, un conjunto de herramientas hardware y software que permitan llevar a cabo el proyecto con la mayor comodidad y precisión posible.

La correcta elección de un entorno de trabajo adecuado es fundamental a la hora de abordar cualquier tipo de proyecto, ya que el éxito o fracaso, o al menos la eficiencia del proceso de desarrollo del mismo, va a depender en gran medida de dicho entorno utilizado.
El apartado actual presenta el entorno de trabajo utilizado para la realización de este \gls{TFM}.

\section{Git y GitHub}

Los sistemas de control de versiones son programas cuyo principal objetivo es controlar los cambios producidos en el desarrollo de cualquier tipo de software, permitiendo conocer el estado actual de un proyecto, las personas que intervinieron en ellos, etc. Además, un buen control de versiones es tarea fundamental para la administración de un proyecto de desarrollo de software en general\cite{alcazar2014}. Git es uno de los sistemas de control de versiones más populares entre los desarrolladores, es gratuito, open source, rápido y eficiente, aunque gran parte su popularidad es debido a GitHub(\autoref{jenkins-logo}), un excelente servicio de alojamiento de repositorios de software que ofrece un amplio conjunto de características de gran utiilidad para el trabajo en equipo.

\begin{figure}[htbp]
	\centering
	\includegraphics[width=0.80\linewidth]
	{entorno/figuras/github.png}
	\caption{Logotipo de GitHub}
	\label{github-logo}
\end{figure}

A continuación se muestran algunas de las características que han llevado a GitHub a ser tan valorado entre los desarrolladores\cite{quintana2015}:

\begin{itemize}
	\item Permite versionar el código, es decir, guardar en determinado momento los cambios realizados sobre un archivo o conjunto de archivos con la oportunidad de tener acceso al historial de cambios al completo, bien para regresar a alguna de las versiones anteriores o bien para poder realizar comparaciones entre ellas.
	\item Gracias a la gran cantidad de repositorios de \gls{SW} públicos que alberga, es posible leer, estudiar y aprender de el código creado por miles de desarrolladores en el mundo, permitiendo incluso la oportunidad de adaptarlos a las necesidades propias de cada desarrollador, sin alterar el original y realizando una copia o fork\footnote{Copia exacta en crudo del repositorio original que podrá ser utilizada como un repositorio git cualquiera} de este.
	\item Tras haber realizado un fork de un proyecto y haber realizado algunos ajuste, introducido alguna mejora o arreglado algún problema que este pudiera contener, es posible integrar los cambios realizados al proyecto original (previa supervisión de su propietario, administrador o alguno de sus colaboradores), por lo que un repositorio puede llegar a ser construido mediante la contribución una gran comunidad de desarrolladores.
	\item GitHub posee un sistema propio de notificaciones con el que poder estar informado de lo que ocurre en torno a un repositorio concreto, ya sea privado a la compañía o público a la comunidad.
	\item 
\end{itemize}



6. Visor de código

GitHub posee un estupendo visor de código mediante el cual, a través del navegador, podremos consultar en cualquier instante el contenido de archivo determinado, con la sintaxis correspondiente a el lenguaje en el que esté escrito. Este navegador es realmente rápido, y gracias a él podremos hacer pequeñas consultas o copiar porciones de código sin necesidad de bajarse todo el repositorio.

7. Mostrar tus conocimientos

Con Github puedes mostrar tus habilidades como desarrollador(a), puesto que es el código escrito en los archivos, donde reposa el resultado del proceso del desarrollo de software. Al compartir tu cuenta de Github con tu potencial empleador o cliente, este podrá ver la calidad del código que escribes a través de los proyectos públicos que estén en tu cuenta. Todos los proyectos que se escriben para ejecutar una idea, aprender un nuevo lenguaje o tecnología son válidos al momento de exhibir tus conocimientos, así que no dudes en publicarlo en tu cuenta. Como complemento a lo anterior, con Github Pages puedes crear incluso una página como esta que te sirva como portafolio, en la cual puedes escribir sobre ti, los conocimientos que posees o poner enlaces de los proyectos en los que has participado.

8. Registro de incidencias

Cada proyecto creado en Github incluye un sistema de seguimiento de problemas, del estilo sistema de tickets, este permite a los miembros de tu equipo (o a cualquier usuario de GitHub si tu repositorio es público) abrir un ticket escribiendo en este los detalles un problema que tenga con tu software o una sugerencia sobre una función que le gustaría que fuera implementada.

9. Compatibilidad

Github es una plataforma web, por tanto es independiente del sistema operativo que utilices, y además Git que es la herramient que si requiere instalación es compatible con todos los sistemas; Linux, OSX y Windows.

10. Precio

Github, es completamente gratis e ilimitado para proyectos públicos, es decir que todos podrán ver el código que estos contienen (aunque tu siempre tendrás el control sobre quien subirá cambios), sin embargo si deseas puedes tener proyectos privados adquiriendo uno de planes que ofrece, los cuales van desde 7 a 50 dólares mensuales, permitiendo crear 5 y 50 repositorios privados respectivamente.



\section{Bundle Audit}



\section{NSP NodeJS}



\section{Clair (CoreOS)}



\section{Docker}



\section{Slack}



\section{Jenkins}

Jenkins (\autoref{jenkins-logo}) es una herramienta autónoma de código abierto que puede utilizarse para automatizar todo tipo de tareas, como la construcción, prueba y despliegue de software. Jenkins puede ser instalado a través de paquetes de sistemas nativos, Docker, o incluso puede ser ejecutado de manera independiente en cualquier máquina con el entorno de ejecución de Java instalado\cite{jenkins2017}.


\begin{figure}[htbp]
	\centering
	\includegraphics[width=0.80\linewidth]
	{entorno/figuras/jenkins.png}
	\caption{\gls{IC} con Jenkins}
	\label{jenkins-logo}
\end{figure}


Jenkins posee, entre otras, las siguientes ventajas:

\begin{itemize}
	\item \textbf{Continuous Integration and Continuous Delivery}: As an extensible automation server, Jenkins can be used as a simple CI server or turned into the continuous delivery hub for any project.
	\item \textbf{Easy installation}: Jenkins is a self-contained Java-based program, ready to run out-of-the-box, with packages for Windows, Mac OS X and other Unix-like operating systems.
	\item \textbf{Easy configuration}: Jenkins can be easily set up and configured via its web interface, which includes on-the-fly error checks and built-in help.
	\item \textbf{Plugins}: With hundreds of plugins in the Update Center, Jenkins integrates with practically every tool in the continuous integration and continuous delivery toolchain. If a plugin does not exist, you can code it and share with the community.
	\item \textbf{Extensible}: Jenkins can be extended via its plugin architecture, providing nearly infinite possibilities for what Jenkins can do.
	\item \textbf{Distributed}: Jenkins can easily distribute work across multiple machines, helping drive builds, tests and deployments across multiple platforms faster.
	\item It is an open source tool with great community support.
	\item It provides continuous integration pipeline support for establishing software development life cycle work flow for your application.
	\item It also provides support for scheduled builds \& automation test execution.
	\item You can configure Jenkins to pull code from a version control server like GitHub, BitBucket etc. whenever a commit is made.
	\item It can execute bash scripts, shell scripts, ANT and Maven Targets.
	\item It can be used to Publish results and send email notifications.
\end{itemize}


\TODO{Pipeline - A user-defined model of a continuous delivery pipeline, for more read the Pipeline chapter in this handbook.\\Debo poner también la imagen de una pipeline de trabajo con verdes y rojos.}

\endinput
	
	%!TEX root =../MemoriaTFM.tex
%El anterior comando permite compilar este documento llamando al documento raíz
\chapter{Desarrollo de la solución}\label{chp-04}
\epigraph{Epígrafe.}{Autor, año}

\lettrine[lraise=-0.1, lines=2, loversize=0.2]{A}{partado} Desarrollo de la solución. 




\endinput

	
	%!TEX root =../MemoriaTFM.tex
%El anterior comando permite compilar este documento llamando al documento raíz
\chapter{Conclusiones}\label{chp-05}
\epigraph{Security is not a line in the sand. Protecting your business, customers, citizens’ data, should be always your number one priority}{Dr. Werner Vogels, 2017\\CTO at Amazon.com}

\lettrine[lraise=-0.1, lines=2, loversize=0.2]{E}{l} apartado actual, tras haber recorrido en la memoria de este \gls{TFM} "\eltitulo" toda la información relativa al proyecto, está dedicado a presentar los objetivos alcanzados y a descubrir algunas de las posibles líneas futuras de trabajo en torno al mismo. 

El conjunto de aplicaciones aportadas a las herramientas y entorno utilizados surge ante la necesidad de las empresa moderna de vigilar de una manera no intrusiva los posibles agujeros y vulnerabilidades de seguridad que contengan las aplicaciones desarrolladas por ellas y los sistemas donde son desplegados. Esta tarea viene requiriendo un esfuerzo excesivo de parte de aquellas personas que se preocupan por mantener los sistemas lo más protegidos posible. 

Uno de los focos de atención en este proceso son las vulnerabilidades en el producto ofertado por la compañía, que pueden estar originadas por la multitud de dependencias de código que va a requerir dicha aplicación para su funcionamiento, así como la inmensa cantidad de imágenes de diferente naturaleza que se despliegan en forma de contenedor en los entornos que dan soporte a las aplicaciones que sustentan el negocio.

Siguiendo esta línea comentada, se decide aportar al proceso de Integración Continua de empresas que ofrecen \gls{SaaS} un sencillo mecanismo de despliegue de contenedores con un conjunto de herramientas y utilidades que van a permitir poder realizar tareas de análisis estático de manera pasiva y como parte del proceso de trabajo natural de la compañía, analizando las posibles posibles vulnerabilidades en las imágenes de contenedores docker y en las depencias  de código utilizado, advirtiendo en el caso de que sean encontradas y aportando informes con los que afrontar la solución al problema.

Por este motivo se puede concluir que, tras el desarrollo de los apartados aquí contenidos, una vez estudiadas las herramientas herramientas aportadas, junto a la posibilidad de despliegue con trabajos automatizados y mecanismo de notificación, se cumple con el objetivo perseguido en la realización de este \gls{TFM}.

Sin embargo, como es trivial en todo proceso tecnológico, la solución aquí entregada no es estática e inamovible y se presta a ser mejorada, además de a mantener los resultados obtenidos actualizados en el tiempo.

Muchas de las tareas y líneas de avance que este nuevo reto contempla aún son desconocidas, ya que van a surgir en la utilización por parte de los usuarios de esta solución. Otras, en cambio, pueden ser establecidas desde el momento actual. A continuación se presentan algunas de las tareas que conforman la línea futura de trabajo de esta aplicación, identificadas durante la realización del proyecto y la redacción de esta memoria:

\begin{itemize}
	\item \textbf{Aumentar el número de lenguajes de programación contemplados en los análisis estáticos}: existen múltiples herramientas en la actualidad de características similares a las aquí presentadas que cubren un amplio abanico de lenguajes de programación diferentes. Implementar un conjunto mayor de dichas herramientas como parte del conjunto actual es una notable mejora a lo aquí presentado.
	\item \textbf{Preparación del entorno para el despliegue en producción}: Contemplando el escenario en que las comunicaciones se realicen con contraseñas encriptadas y utilizando mecanismos de certificados SSL/TLS.
	\item \textbf{Ampliar los repositorios alcanzados}: Aumentando el número de parámetros que los trabajos de Jenkins reciben a la entrada se pueden variar la rama de código a la que se le realiza el análisis, el usuario que almacena el repositorio en su cuenta de GitHub o ambos, ampliando considerablemente el número de posibles repositorios objetivos.  
\end{itemize}

El resultado obtenido a partir del presente \gls{TFM} es un producto con utilidad tangible, que ya está siendo utilizado en los sistemas de Workshare Inc., proporcionando valor añadido a la actividad en materias de Seguridad de la Información y las Comunicaciones que allí se realiza. Este aspecto es elemento clave a la hora de diferenciar el proyecto realizado, ya que existen múltiples trabajos realizados que nunca llegan a superar la barrera de estudios teóricos, para convertirse en un producto o herramienta final en completo funcionamiento, lo que sin duda es una enorme satisfacción para su autor.

\endinput

	
	%!TEX root =../MemoriaTFM.tex
%El anterior comando permite compilar este documento llamando al documento raíz
\begin{appendices}
	\chapter{Repositorio de GitHub}\label{ap-01}
	\epigraph{Share your knowledge. It’s a way to achieve immortality.}{Dalai Lama\\high lama of Tibetan Buddhism}
	\TODO{Aquí voy a explicar que el TFM se encuentra en el repositorio, como accederlo y donce ver los codigos, con imagen incluida.}
		
\end{appendices}
\endinput
	
	%:Empieza todo lo que no constituye el cuerpo en si del libro.
	\backmatter
	
	%:Indice de figuras.
	\cleardoublepage
	\phantomsection
	
	%:Para añadir una línea en blanco en el TOC y separar esta lista
	\addtocontents{toc}{\protect\mbox{}\protect\hspace*{0pt}\par}
	\addcontentsline{toc}{listasb}{\listfigurename}
	\pagestyle{especial}
	\listoffigures
	
	%:Indice de tablas, coméntese las siguientes líneas si no se desea
	%:\cleardoublepage
	%:\phantomsection
	%:\addcontentsline{toc}{listasb}{\listtablename}
	%:\pagestyle{especial}
	%:\listoftables
	
	%:Indice de Código
	\cleardoublepage
	\phantomsection
	\addcontentsline{toc}{listasb}{\lstlistlistingname}
	\pagestyle{especial}
	\lstlistoflistings
	
	%:Bibliografía con biblatex y biber
	\cleardoublepage
	\phantomsection
	\addcontentsline{toc}{listasb}{\refname}
	\pagestyle{especial}
	%BIBER
	%\printbibliography[heading=etsi]z
	%BIBTEX
	%\bibliographystyle{IEEEtran}
	\bibliographystyle{amsplain} %flexbib amsplain alpha
	%:Fichero con la bibliografía, BIBTEX
	\bibliography{bibliografia}
	
	
	%:Acrónimos
	\cleardoublepage
	\phantomsection
%	\addcontentsline{toc}{listasb}{\glossaryname}
%	\chaptermark{\glossaryname}
	\printglossary[type=\acronymtype,title=Glosario]
	
\end{document}