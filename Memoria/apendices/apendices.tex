%!TEX root =../MemoriaTFM.tex
%El anterior comando permite compilar este documento llamando al documento raíz
\begin{appendices}
	\chapter{Repositorio de GitHub}\label{ap-01}
	\epigraph{Share your knowledge. It’s a way to achieve immortality.}{Dalai Lama\\high lama of Tibetan Buddhism}
	
	\lettrine[lraise=-0.1, lines=2, loversize=0.2]{T}{odo} el contenido de la memoria actual para el Trabajo fin de Máster \gls{TFM} "\eltitulo", así como los archivos y la información necesaria para desplegar el entorno de pruebas realizado se encuentran a disposición pública en los repositorios de Github, con el único objetivo de seguir avanzando en el desarrollo de la idea y que permanezca disponible a cualquier usuario que pudiera beneficiarse de la misma. La \autoref{github} muestra el repositorio comentado.
	
	\begin{figure}[htbp]
		\centering
		\includegraphics[width=1.0\linewidth]
		{apendices/figuras/github.png}
		\caption{https://github.com/EleazarWorkshare}
		\label{github}
	\end{figure}
	
\end{appendices}

Por supuesto, cualquier contribución será bien recibida.
\endinput